\documentclass{hertieteaching}
\usepackage{cancel}

\title{Introduction: Workflow}

\begin{document}

\maketitle

\begin{frame}{Taming chaos}

In data analysis there are two sorts of \textit{surprises} and two sorts of \textit{cognitive stress}
\begin{enumerate}
  \item Analytical (often good)
  \item Infrastructural (almost always bad)
\end{enumerate}
Analytical surprise is when you learn something from or about the data 

Infrastructural surprise is when you discover you that 
\begin{itemize}
  \item you can't find what you did before
  \item the analysis code breaks
  \item the report doesn't compile
  \item the co-author can't run your code
\end{itemize}

Data analysis can be stressful, workflow is there to ensure the only the \textit{right kind} of stress

\end{frame}

\begin{frame}{Taming chaos}
  
There is a division of labour:
\begin{itemize}
  \item Some languages have strong feeling about how work should flow, e.g. \textsf{Java}, \textsf{c++}
  \item Some do not, e.g. \textsf{python} and \textsf{R}
\end{itemize}
For languages that do not impose themselves, we must use \textit{convention}.

So this lecture is very largely about workflow conventions that we have collectively found to work well 
\begin{itemize}
  \item This is an R-based course, but only some of these conventions are R-specific. Feel free to export the rest to your other projects
\end{itemize}


\end{frame}


\begin{frame}{Essentials}

You should \textit{always} think in terms of projects
  
A project is a self-contained unit of data science work that can be 
\begin{itemize}
  \item shared
  \item recreated by others
  \item packaged
  \item mothballed
\end{itemize}
It contains
\begin{itemize}
  \item Content, e.g. raw data, processed data, scripts, functions, documents
  \item Metadata: information about tools for running it, e.g. required libraries, compilers  
\end{itemize}

\end{frame}
\begin{frame}{Essentials: Structure}

It is easiest to use your \textit{file system} to maintain this discreteness, and plain text files to record the metadata

\begin{block}{Plain text ideology}
See \textcite{Healy2019} for the advantages of working as far as possible with human-readable `plain text' formats over proprietary and binary ones
\end{block}  

For R projects
\begin{itemize}
  \item Projects are folders/directories
  \item Metadata is the \textsf{.Rproj} files, perhaps augmented with the output of \textsf{renv}
\end{itemize}



\end{frame}
\begin{frame}{Essentials: Structure}

The ideal:  
\begin{itemize}
  \item One folder contains everything inside it
  \item All internal paths are relative and point sideways or downwards
  \item Can be relocated without problem
  \item Put `product'-generating files at the top level
\end{itemize}

Paths and working directories can confuse people (not you of course), so these get their own conventions

\end{frame}
\begin{frame}[fragile]{Essentials: Good paths}

Good paths: 
\begin{itemize}
  \item \texttt{"preprocessing.R"}, 
  \item \texttt{"figures/perpetual-motion.png"}
\end{itemize}
Invariant to moving or sharing the project folder

\end{frame}
\begin{frame}[fragile]{Essentials: Bad paths}


Bad paths:
\begin{itemize}
  \item \texttt{"../code/funcs.R"}
  \item \texttt{"/Users/me/data/thing.sav"}
\end{itemize}

The first is confusing but not terrible. The second will not work if you share your project

~\\
But sometimes keeping data elsewhere \textit{is} unavoidable. Then
\begin{verbatim}
DATA <- file.path(Sys.getenv("HOME"), "data/thing.sav")
\end{verbatim}
can be handy. 
\begin{itemize}
  \item Hint: do all this at the \textit{very top} of whatever product you are working on
\end{itemize}

\end{frame}

\begin{frame}[fragile]{Essentials: Bonus}

You'll never have to write 
\begin{verbatim}
setwd("/Users/me/whereever/I/left/the/project")
\end{verbatim}
which is good because that won't travel either.

~\\
Just let it go\ldots

\end{frame}
\begin{frame}{Essentials: The working directory}
  
Set it manually once per session (\textsf{Session $>$ Set Working Directory $>$ Choose Directory}). Then all your good paths will ``just work''

Better yet, have the metadata set it for you
\begin{itemize}
  \item Open your session by opening (choosing, clicking on) ~\texttt{myproject.Rproj}
  \item Then you'll get the path set for you
\end{itemize}

Bonus bonus: You won't look like an amateur\ldots

\end{frame}
\begin{frame}{Essentials: Metadata}

With external dependencies comes chaos.

Early in a project, you can be fairly loose about libraries and dependencies

As soon as you share it, you need to care\ldots

For regular supporting tools, e.g. latex, use the \texttt{.Rproj} file to record your choices

For R libraries:
\begin{itemize}
  \item lightweight solution: Put all your \texttt{library} calls at the top of the product (not in any files it calls).
  Rstudio will pick up uninstalled libraries and ask if you want to install them
  \item heayweight solution: \texttt{renv} and the tools on the \href{https://cran.r-project.org/web/views/ReproducibleResearch.html}{Reproducibility Task View}
\end{itemize}

For system dependencies, e.g. \textsf{C,C++} libraries, compilers, python versions: 
\begin{itemize}
  \item This is \textit{just hard}. `Some setup required'
\end{itemize}



\end{frame}
\begin{frame}{Essentials: Data}

Don't store \textit{derived data} or \textit{intermediate state} without the recipe

\pause

What is derived data?
\begin{itemize}
  \item Wrangled \textit{raw} data, e.g. the processed survey file with just your questions in it
  \item A thousand web pages scraped from a website
  \item Graphs and tables
\end{itemize}
(Actually the web is a partial exception to this rule: but you still need to keep the download code)

What is derived state? The objects in your local environment
\begin{itemize}
  \item things you made from the Console, e.g. \textsf{.RData}
  \item intermediate computations, e.g. compiled \textsf{Stan} models or \textsf{Rcpp} functions
\end{itemize}

Why?

\end{frame}
\begin{frame}{Essentials: Entry points}

A data science project can, and usually does, have several distinct products
\begin{itemize}
  \item The analysis itself
  \item Graphs and tables
  \item An intermediate report of the analysis
  \item An academic paper
  \item Overhead slides
\end{itemize}

These products should also project \textit{entry points} 
\begin{itemize}
  \item Q: ``So, where are we on this project?''
  \item A: [Knits report]
\end{itemize}

Ideally you should be able to generate each of them in one step


\end{frame}
\begin{frame}{Essentials: Analysis entry points}

\texttt{script.R} should do \textit{everything} when \texttt{source}d. Assert the dependencies, source other scripts, there

Otherwise you have to remember to do things in the right order

\begin{block}{Scripting}
Have a script, don't be a script
\end{block}

\end{frame}
\begin{frame}{Essentials: Experimentation}

Only \textit{one} script (per product), ever other bit of code should contain \textit{functions, not commands}
\begin{itemize}
  \item When something goes wrong you want to debug functions, not dig up scripts
\end{itemize}

\begin{block}{Debugger}
Make friends with \texttt{debug}, \texttt{undebug}, and the debugger
\end{block}

Why? \textit{Damage limitation}
\begin{itemize}
  \item Functions (when written right) restrict the information being used to their parameters
\end{itemize}



Note: Sometimes they also hold information about the external environment where they were defined too, a.k.a. `enclosures'. Try not to depend on that sort of thing\ldots 

\end{frame}

\begin{frame}{Essentials: Experimentation}

Keep the little experiments in the Console, or in a scratch file.
\begin{itemize}
  \item When you like the code, add it to the main script.
  \item Maybe don't put that stuff in a huge code block in an \texttt{.Rmd} file
\end{itemize}

\end{frame}


\begin{frame}{Essentials: Figures and tables}

Have a folder for figures and tables
\begin{itemize}
  \item Be ready to overwrite everything there
\end{itemize}
One exception: \textsf{RMarkdown} will generate its own transient figure folder, don't bother to mess with that one

\textit{Never} write data tables by hand. Just like you wouldn't write a bibliography by hand (would you?)

Choose a package that formats tables nicely and learn how to use it
\begin{itemize}
  \item I like \texttt{texreg} and \texttt{xtable} for Latex and \texttt{kable} for Rmarkdown
  \item There are probably better choices, but I don't care
\end{itemize}

\end{frame}

\begin{frame}[fragile]{Essentials: Figures and tables}

This is useful even if you \textit{don't} use \texttt{knitr} to make your documents

For example, in the R script
\begin{verbatim}
options(xtable.floating = FALSE, latex.environments = "center",
        xtable.comment = FALSE, xtable.booktabs = TRUE,
        xtable.caption.placement = "top")
...
print(xtable(tbl), file = "tables/tab2.tex) 
\end{verbatim}
and in my Latex document:
\begin{verbatim}
\input{tables/tab2.tex}
\end{verbatim}
\end{frame}

\begin{frame}[fragile]{Essentials: Documents}
  
Decisions, decisions:
\begin{itemize}
  \item Does the analysis code go in the document? (maybe, if \textit{simple} enough)
  \item Does the table and figure code go in the document? (maybe, if it doesn't take too \textit{long} to run) 
\end{itemize}

The \href{https://bookdown.org/yihui/rmarkdown-cookbook/cache.html}{RMarkdown cookbook} has a lot of good advice, even if you don't write in \textsf{RMarkdown}
\begin{itemize}
  \item Compile early, and compile often\ldots 
\end{itemize}
  
Aside: Got a beautiful slide template? e.g. \texttt{myslides.cls} or a pandoc template \texttt{myslides.tex}.
\begin{itemize}
  \item Put that stuff in a folder if you want your friends to use it
\end{itemize}

  
\end{frame}

\begin{frame}[fragile]{Essentials: Friends and collaborators}

It's often tempting to set up a project assuming that you will be the only person working on it, e.g. as homework

That's almost never true. 
\begin{itemize}
  \item Coauthors happen to the best of us
  \item Even if not, there's someone else who you always have to keep happy
\end{itemize}

\pause

Future-you
\end{frame}
\begin{frame}[fragile]{Essentials: Friends and collaborators}

Future-you is really the one you do organize your projects for

They are who you use version control for (more on that next week)

Most importantly, they are who will enjoy the fruits of your data science labour, or have to fight back your chaos.
\begin{itemize}
  \item So be kind to Future-you. Establish a good workflow. You'll thank yourself later.
\end{itemize}

\end{frame}

\begin{frame}[allowframebreaks]
\frametitle{References}
\printbibliography
\end{frame}

\end{document}


