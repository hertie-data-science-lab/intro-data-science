\documentclass{hertieteaching}
\usepackage{cancel}
\usepackage{hyperref}
\graphicspath{{.}{../

\title{Text (as Data)}

\begin{document}

\maketitle


\begin{frame}{Why text?}

Applications:
\begin{itemize}
  \item Agenda measurement \cite[e.g.][]{Grimmer.etal2011}
  \item Framing studies \cite[e.g.][]{Gamson.Modigliani1989}
  \item Authorship attribution \cite[e.g.][]{Mosteller.Wallace1963}
  \item Bias measurement \cite[e.g.][]{Caliskan.etal2017}
  \item Policy preference estimation \parencite[e.g.][]{Laver.etal2003}
\end{itemize}

\end{frame}
\begin{frame}{Why text?}
  
Text data is 
\begin{itemize}
  \item ubiquitous
  \item easily collectable
  \item informative, even where other behaviour is not
\end{itemize}

\pause

Nevertheless also
\begin{itemize}
  \item awkward to work with
  \item often strategically generated (or \textit{not} generated)
  \item difficult to compare across genres, languages, institutions
\end{itemize}

\end{frame}


\begin{frame}{Not (just) nlp}

Overlapping NLP tasks
\begin{itemize}
  \item Segmentation / tokenization: Locating words and sentences
  \item Part of Speech (POS) tagging: Associating grammatical
         roles with words (noun, verb, determiner, preposition, etc.)
  \item Parsing: grammatical structure from sentences
\end{itemize}
Distinctly NLP tasks
\begin{itemize}
  \item Named Entity Recognition (NER): Identifying people, places, and things
  \item Information Extraction (IE): Extracting `facts' (who did what to whom, when)
\end{itemize}

%Tools:
%
%- [Spacy](https://spacy.io) (Python) via `{spacyr}`,
%  [Stanford NLP tools](https://nlp.stanford.edu/software/),
%  (Java) via `{CoreNLP}` (or `{cleanNLP}` which wraps both).
%  Also [OpenNLP](http://opennlp.apache.org/) (Java) via
%  `{openNLP}`.
%

\end{frame}

\begin{frame}{Difficult data}

The Zipf-Mandelbrot law \parencite{Zipf1932,Mandelbrot1966}
$$
C(W_i) \propto 1/{\text{rank}(W_i)^\alpha}
$$
where $\text{rank}(.)$ is the
frequency \textit{rank} of a word in the vocabulary and
$\alpha\approx 1$

(This is a Pareto distribution in disguise)

\pause

\bigskip
Intuition: 
\begin{itemize}
  \item Most words occur in \textit{very} low frequencies, while a handful dominate
\end{itemize}


\end{frame}

\begin{frame}{Difficult at all scales}

\begin{columns}[T,onlytextwidth]
\column{0.45\textwidth}

\centerline{\includegraphics[scale=0.7]{pics/allscalesorig}}

%\column{0.1\textwidth}
\column{0.45\textwidth}

\centerline{\includegraphics[scale=0.7]{pics/allscaleslog}}

\end{columns}

This is a \textit{power law} relationship: see also \textcite{Chater.Brown1999} on scale invariance.

\end{frame}

\begin{frame}{Types and Tokens}

\begin{columns}[T,onlytextwidth]
\column{0.5\textwidth}

More generally: the Heaps-Herdan Law states that the number of
word types appearing for the first time after n tokens is
$$
D(n) = K n^\beta
$$
where $K$ is between 10 and 100 and $\beta \approx 0.5$ for English.

~\\

(All the party manifestos shown here)

\column{0.05\textwidth}
\column{0.45\textwidth}

\centerline{\includegraphics[scale=0.7]{pics/heaps}}

\end{columns}

\end{frame}

\begin{frame}{Frequency and interestingness}

Frequency is inversely proportional to substantive interestingness

~\\
\begin{columns}[T,onlytextwidth]
\column{0.33\textwidth}

\begin{center}
% latex table generated in R 4.0.2 by xtable 1.8-4 package
% Tue Sep  8 12:53:28 2020
\begin{table}[ht]
\centering
\begin{tabular}{rlr}
  \hline
 & Word & Freq. \\ 
  \hline
1 & the & 21939 \\ 
  2 & and & 15747 \\ 
  3 & to & 15347 \\ 
  4 & of & 10850 \\ 
  5 & we & 7943 \\ 
  6 & will & 7930 \\ 
   \hline
\end{tabular}
\end{table}


~\\
Top 10
\end{center}
\column{0.33\textwidth}


\begin{center}

% latex table generated in R 4.0.2 by xtable 1.8-4 package
% Tue Sep  8 12:53:28 2020
\begin{table}[ht]
\centering
\begin{tabular}{rlr}
  \hline
 & Word & Freq. \\ 
  \hline
16078 & 1.83 & 1 \\ 
  16079 & 2.20 & 1 \\ 
  16080 & 1.35 & 1 \\ 
  16081 & 33.34 & 1 \\ 
  16082 & 1.71 & 1 \\ 
  16083 & rigination & 1 \\ 
   \hline
\end{tabular}
\end{table}


~\\
Bottom ten
\end{center}

\column{0.33\textwidth}
\begin{center}
% latex table generated in R 4.0.2 by xtable 1.8-4 package
% Tue Sep  8 12:53:29 2020
\begin{table}[ht]
\centering
\begin{tabular}{rlr}
  \hline
 & Word & Freq. \\ 
  \hline
20 & people & 1929 \\ 
  26 & new & 1507 \\ 
  27 & government & 1493 \\ 
  33 & support & 1212 \\ 
  34 & work & 1143 \\ 
  36 & uk & 1058 \\ 
   \hline
\end{tabular}
\end{table}


~\\
Top ten minus \textit{stopwords}
\end{center}

\end{columns}

\end{frame}

\begin{frame}{Stopwords}

Stopwords are a list of words that we think won't be worth keeping track of and will only get in the way of analysis
\begin{itemize}
  \item Not outliers (they're usually the most common!)
  \item Like speech tics and pauses in a speech transcript: `not worth transcribing'
\end{itemize}

%\centerline{\decorativeflower}
\bigskip
Removing \textit{stopwords}, while standard in computer science, is not necessarily better\ldots

Example:
\begin{itemize}
\item Standard collections contain, `him', `his', `her' and `she'.
\item Words you'd want to \textit{keep} when analyzing an abortion debates.
\end{itemize}

%\centerline{\decorativeflower}

Reminder: `Preprocessing' steps like this are model fitting in disguise
\end{frame}

\begin{frame}{Bags of words}

One big distinguishing feature of text as data approaches from NLP is the willingness to make \textit{bag of words} assumptions

Formally, the BOW assumption says: words occurrences are \textit{exchangeable}, approximately:
\begin{itemize}
  \item Document \textit{content} does not depend on the order of the words
\end{itemize}
So \parencite{deFinetti2008} we can model words as independently generated, \textit{conditional on} a message $\theta$
\begin{align*}
P(\text{``unemployment is socially corrosive''}) & = P(\{\text{corrosive}, \text{unemployment}, \text{socially}, \text{is}\})\\
 & = \int\- \prod^{\substack{\{\text{corrosive, is}\\
   \text{unemployment}\\\text{socially}\}}}_w \-\-P(W=w \mid \theta) P(\theta) \text{d}\theta
\end{align*}

Clearly this is a better assumption in some genres than others\ldots

\end{frame}

\begin{frame}{The data and the message}

Bags of words are \textit{contingency tables} $C$, or term-document / document-term / document-feature matrices, in the lingo

\begin{center}
{\small
\begin{tabular}{rllllllllll}\toprule
      & corrosive & is & unemployment & socially & a & & \\ \midrule
doc 1 & 1     & 1  & 2 & 0    & 2    & \ldots & \textcolor{gray}{$\theta_{doc1}$} \\
doc 2 & 0     & 0  & 1 & 1    & 12    & \ldots & \textcolor{gray}{$\theta_\text{doc2}$} \\ \midrule
      & \textcolor{gray}{$\beta_\text{corrosive}$}
      & \textcolor{gray}{$\beta_\text{is}$}
      & \textcolor{gray}{$\beta_\text{unemployment}$}
      & \textcolor{gray}{$\beta_\text{socially}$}
      & \textcolor{gray}{$\beta_\text{a}$} \\ \bottomrule
\end{tabular}
}
\end{center}
What is $\theta$?
\begin{itemize}
  \item A sample of words from a \textit{single} topic (category, subject, etc.): document classification
  \item A mixed bag of \textit{topics} (categories, emphases, etc.) in particular proportions: topic models and dictionary-based content analysis
  \item A sample of words from a single \textit{position} in some space: scaling models
\end{itemize}
  
\end{frame}

\begin{frame}{Document classification}

\bigskip
\centerline{\includegraphics[scale=0.4]{pics/evans-document-classification} } 

From \cite{Evans.etal2007}.
\end{frame}
\begin{frame}{Document classification}

Evan et al. try to distinguish Amicus briefs in favour of the defendants or the plaintiffs in two US affirmative action cases
\begin{itemize}
  \item Classifier: `Naive Bayes'
\end{itemize}

This is a \textit{generative} classifier, meaning it tries to learn how words would be generated if you were supporting the defendant vs the plaintiff
\begin{itemize}
  \item $P(\{W\} \mid Y =\text{plaintiffs}) = \prod_w^{\{W\}} P(W=w \mid Y =\text{plaintiffs})$
  \item $P(\{W\} \mid Y =\text{defendants}) = \prod_w^{\{W\}} P(W=w \mid Y =\text{defendants})$
\end{itemize}
then \textit{reverses} this using Bayes Theorem to infer
\begin{itemize}
  \item Supports the plaintiff: $P(Y =\text{plaintiffs} \mid \{W\})$ 
  \item Supports the defendants: $1-P(Y =\text{plaintiffs} \mid \{W\})$
\end{itemize}

\end{frame}
\begin{frame}{Document classification}

This inefficient: 
\begin{itemize}
  \item some words are used at equal rates by both sides, so are useless for distinguishing them
  \item but they're in the mix anyway, even if just noise
\end{itemize}
But conveniently, we get a vocabulary analysis as a side product, e.g.
$$
\frac{P(\text{`benign'} \mid Y =\text{plaintiffs})}{P(\text{`benign'} \mid Y =\text{defendants})}
$$
Intuition: If this is large then using `benign' \textit{distinguishes} the plaintiffs
  
\end{frame}
\begin{frame}{Document classification}

\bigskip
\centerline{\includegraphics[scale=0.25]{pics/evansetal1} \includegraphics[scale=0.3]{pics/evansetal2} }  

\bigskip
In the dictionary, `benign' has a broadly positive valence. In this situation it is quite loaded in favour of the plaintiffs.
\end{frame}

\begin{frame}{Vocabulary contrasts}

Alternatively, these comparisons may be the focus of a text analysis, not a byproduct
\begin{itemize}
  \item The \texttt{quanteda} package calls these differences `keyness'
\end{itemize}
For example, here's some comparisons of the words that discriminate 2016 Trump on Twitter, depending on whether the source of the tweet is an iPhone or an Android phone
\begin{itemize}
  \item Theory: staff used iPhones and posted campaign messages to his account, he has an Android
\end{itemize}
	
\end{frame}
\begin{frame}{ANALYZE TWITTER DATA. VOTE!}

\centerline{\includegraphics[scale=0.5]{pics/trumptweets}}

\end{frame}





\begin{frame}{Dictionaries and topic models}

Previously we assumed that documents expressed only one thing, eg. support for the plaintiffs

What if we believed that the message was in the mixture of topics it contained?

Two approaches:
\begin{itemize}
  \item Confirmatory, and manual: build a content analysis dictionary
  \item Exploratory (mostly), and automated: fit a topic model
\end{itemize}
  
\end{frame}

\begin{frame}{Topics}

\medskip
\centerline{\includegraphics[scale=0.3]{pics/topics2}}
From \textcite{Blei.etal2003}

\end{frame}

\begin{frame}{Topics}

\medskip
\centerline{\includegraphics[scale=0.4]{pics/topic-words}}
From \textcite{Quinn.etal2006}\\
Note: only the top most probable words are shown and topic labels are manually assigned.

\end{frame}

\begin{frame}{Topics}

\centerline{\includegraphics[scale=0.35]{pics/defence-topic-quinn}}

From \textcite{Quinn.etal2006}

\end{frame}

\begin{frame}{Topic model training}

Topic models can be quite hard and time consuming to estimate. We start with
\begin{itemize}
  \item A term document matrix $C$ (the contingency table)
  \item A belief about the number of topics $K$
\end{itemize}
and try to learn
\begin{itemize}
  \item a topic label $Z=3$ for each word
  \item a `dictionary' $\beta_{w3} = P(W=w \mid Z=3)$ for every $w$ and every topic
  \item the proportion of each topic, e.g. $\theta_3$ in each document
\end{itemize}
These are all coupled, and all unknown.

We can help a bit with hyperparameters that give the model a `prior' over $\theta$ and/or over $\beta$

\end{frame}

\begin{frame}{Interpreting topic models}

Ideally we'd like to be able to say: ``make this one about defense''

Unfortunately, that level of high level control is an unsolved problem

We can only -- after the fact -- label the topics, and hope some are topics we want.


\end{frame}
\begin{frame}{Topic model topics}

Are they good, these topics? Ironically
\begin{itemize}
  \item the better the \textit{statistical} properties of the model the less interpretable it tends to be \parencite{Chang.etal2009}
  \item Clearly we're missing something with the model structure\ldots
\end{itemize}

\end{frame}

\begin{frame}{Explaining topic prevalence}

Often we want to both measure and explain the prevalence of topic mentions, e.g. the effects of a Japanese electoral reform \parencite{Catalinac2018}

\centerline{\includegraphics[scale=0.4]{pics/jpcontent}}
	
\end{frame}


\begin{frame}{Structural topic model}

Topic models usually end by presenting us with $\hat{\theta}$ for each document and a dictionary of $\beta$s

If we like some of the topics, we might want to know how they vary with external information, e.g.
\begin{itemize}
  \item How does rate of topic 3, say `defence', change with the party of the speaker?
\end{itemize}
This is a regression model with 
\begin{itemize}
  \item Speaker party indicator $X$ (observed)
  \item proportion of the speech assigned to topic 3 as $Y^*$ (inferred, not observed)
  \item Covariates $Z$, e.g. committee membership, date, etc. (observed)
\end{itemize}
The \textit{structural topic model} \parencite{Roberts.etal2014} mixes together
\begin{itemize}
  \item fitting the topic model
  \item conditioning its output on the $X$ and $Z$
\end{itemize}
Convenient!


\end{frame}

\begin{frame}{Structural topic model}

Having a topic model allows us to get contrast vocabulary \textit{within topic} too. 

Here's contrasting usage when talking about Guatanamo Bay in a Bush era data set

\centerline{\includegraphics[scale=0.5]{pics/comparisons-stm}}
	
\end{frame}
\begin{frame}{Scaling}

We can also think about document living in some kind of \textit{space} with $\theta$ as the positione.g.
\begin{itemize}
  \item affect, a.k.a. \textit{sentiment analysis}
  \item unidimensional policy preferences
  \item multidimensional ideological position
\end{itemize}

How to place documents in space?
\begin{itemize}
  \item Think of a row in the document term matrix as a vocabulary profile, e.g. by normalize the counts
  \item This is a point in a (very high-dimensional) space
  \item Which has distances to every other document in that space
\end{itemize}
We can, and do, cluster documents this way.

\end{frame}

\begin{frame}{Scaling}

But we can also collapse them down into a smaller space, e.g. one or two dimensions
\begin{itemize}
  \item Often we think they really live there
  \item Sometimes it's just visualization
\end{itemize}

The model is quite simple. If $C_{ij}$ is the number of times the $j$-th word occurs in the $i$-th document, then, in one dimension
$$
\log C_{ij} = \alpha_i + \psi_j + \theta_i\beta_j
$$

\pause

and in more than $K$ (orthogonal) dimensions
$$
\log C_{ij} = \alpha_i + \psi_j + \sum^K_k \theta^{(k)}_i\sigma^{(k)}\beta^{(k)}_j
$$
where $\sigma^{(k)}$ is the importance of that dimension to $C$


\end{frame}

\begin{frame}{Scaling one dimension}

\centerline{\includegraphics[scale=0.35]{pics/fomc-ip}}
Estimated FOMC member ideal points from meeting transcripts  \textcite{Baerg.Lowe2020}

\end{frame}

\begin{frame}{Scaling several}
\vspace{-1em}
\centerline{\includegraphics[scale=0.45]{pics/grey-just-rile}}
Post-1989 German party position from CMP-coded platforms

\end{frame}

\begin{frame}{In each other's space}
\medskip
\centerline{\includegraphics[scale=.35]{pics/cmpcats_us_in_de}}
Link: \href{https://www.nytimes.com/interactive/2019/06/26/opinion/sunday/republican-platform-far-right.html}{the pretty version at the New York Times}

\end{frame}

\begin{frame}{Text (as data)}

Lots of possibilities -- ask about them in class!
	
\end{frame}

\begin{frame}[allowframebreaks]
\frametitle{References}
\printbibliography	
\end{frame}

\end{document}